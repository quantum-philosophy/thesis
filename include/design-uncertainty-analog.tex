In this chapter, we shift our attention from analyzing process variation to
designing with process variation. In other words, instead of analyzing the
original variability, we now study its impact on higher-level characteristics so
that this deteriorating impact can be taken into account in the final design.

\section{Introduction}
\inputsection{introduction}

\section{Motivational Example}
\slab{chaos-example}
\inputsection{example}

\section{Problem Formulation}
\slab{chaos-problem}
\inputsection{problem}

\section{Prior Work}
\slab{chaos-prior}
\inputsection{prior}

\section{Our Solution}
\slab{chaos-solution}
\inputsection{solution}

\section{Uncertainty Analysis}
\slab{chaos-uncertainty-analysis}
\inputsection{uncertainty-analysis}

\section{Transient Analysis}
\slab{chaos-transient-analysis}
\inputsection{transient-analysis}

\section{Dynamic Steady-State Analysis}
\slab{chaos-dynamic-steady-state-analysis}
\inputsection{dynamic-steady-state-analysis}

\section{Reliability Analysis}
\slab{chaos-reliability-analysis}
\inputsection{reliability-analysis}

\section{Reliability Optimization}
\slab{chaos-reliability-optimization}
\inputsection{reliability-optimization}

\section{Illustrative Application 1}
\slab{chaos-application}
\inputsection{application}

\section{Illustrative Application 2}
\inputsection{application-2}

\section{Experimental Results 1}
\slab{chaos-result}
\inputsection{result}

\section{Experimental Results 2}
\inputsection{result-2}

\section{Conclusion}
\slab{chaos-conclusion}
\inputsection{conclusion}
