\chapter{Development with Certainty}

In this chapter, we elaborate on deterministic system-level analysis of computer
systems. The techniques that we discuss here do not take uncertainty into
account. However, these techniques provide a solid foundation for those that do
as we shall describe in the following chapter, \cref{development-uncertainty}.

\section{Introduction}
Due to the rapidly increasing power densities, temperature has evolved into an
acute concern of computer-system designs, and temperature analysis has become an
essential component of design workflows. With this in mind, we focus our
attention on this analysis and its accompanying disciplines.

There are two types of temperature analysis: \one~transient analysis and
\two~steady-state analysis. The first delivers the temperature that the system
at hand transitions over within an arbitrary time interval staring from
arbitrary initial conditions when it is exposed to an arbitrary, both spatially
and temporally, power distribution. The second delivers the temperature that the
system attains and retains once it has reached a thermal equilibrium under a
power distribution that is spatially arbitrary but temporally constant or
repeating.

Steady-state analysis can be further classified into two categories, which have
already been alluded to: \one~static analysis and \two~dynamic analysis. The
first is concerned with temporally constant power distributions, that is, with
those that do not change over time. In this case, temperature distributions do
not change over time either. The second is concerned with temporary arbitrary
power distributions that are periodic, that is, with those that repeat with
certain periods. In this case, temperature distributions change over time with
the same periods as the corresponding repeating power distributions.

The agenda for the rest of this chapter is as follows. In
\sref{general-analysis}, we present the general model that is used in all the
three aforementioned types of temperature analysis. Transient analysis, static
steady-state analysis, and dynamic steady-state analysis are then discussed
separately in \sref{transient-analysis}, \sref{static-steady-state-analysis},
and \sref{dynamic-steady-state-analysis}, respectively. Lastly, the importance
of the last in the context of reliability optimization is covered in
\sref{reliability-optimization}.


\section{General Temperature Analysis}
\slab{general-analysis}
\input{include/development/certainty/general-analysis}

\section{Transient Temperature Analysis}
\slab{transient-analysis}
The goal of transient analysis is to compute the temperature profile \mq that
corresponds to a given power profile \mp by solving
\eref{temperature-model-original}. Suppose for now that the power consumption is
constant: $\vp(t) = \vp$. Then \eref{temperature-differential-original} is a
system of ordinary differential equations that has the following analytical
solution:
\begin{equation} \elab{solution-original}
  \tvs(t) = e^{\tm{A} t} \tvs_0 + \tm{A}^{-1} (e^{\tm{A} t} - \m{I}) \m{C}^{-1} \tm{B} \vp
\end{equation}
where $\tvs_0$ is the initial condition, $\m{I}$ is the identity matrix, and
\[
  \tm{A} = -\m{C}^{-1} \m{G}.
\]
Assume now that the sampling interval \dt of \mp is sufficiently small so that
the power consumption in the interval $[t_i, t_{i + 1})$ can reasonably be
approximated by a constant equal to $\vp_i = \vp(t_i)$. Then the corresponding
\mq can be found by applying the following recurrence derived from
\eref{solution-original}:
\begin{equation} \elab{recurrence-original}
  \tvs_{i} = \tm{E} \tvs_{i - 1} + \tm{F} \vp_i
\end{equation}
for $i = \range{1}{\ns}$ where
\begin{align*}
  & \tvs_0 = 0, \\
  & \tm{E} = e^{\tm{A} \dt}, \text{ and} \\
  & \tm{F} = \tm{A}^{-1}(e^{\tm{A} \dt} - \m{I})\m{C}^{-1} \tm{B}.
\end{align*}
It should be noted that, in order to obtain the actual \mq, the recurrence
should be followed by \eref{temperature-algebraic-original}, which involves two
trivial algebraic operations.

Similar to the observation is made in \cite{thiele2011}, our experiments show
the approach to transient temperature analysis described above provides a
significant performance improvement compared to iterative solutions to ordinary
differential equations such as the Runge--Kutta fourth-order method
\cite{press2007}. However, there is still room for improvement as follows.

Even though the matrices $\tm{E}$ and $\tm{F}$ have to be computed only once,
they necessitate two computationally problematic operations: the matrix
exponential and matrix inverse involving $\tm{A} \in \real^{\nn \times \nn}$,
which is a generic matrix. It is preferable to have a symmetric matrix $\m{A}
\in \real^{\nn \times \nn}$ when these operations are concerned since such a
matrix admits the eigendecomposition \cite{press2007}
\begin{equation} \elab{eigendecomposition}
  \m{A} = \m{U} \m{\Lambda} \transpose{\m{U}}
\end{equation}
where $\m{U}$ is a square matrix of the eigenvectors of $\m{A}$ that are
orthogonal, and $\m{\Lambda} = \diagonal{\lambda_1}{\lambda_\nn}$ is a diagonal
matrix of the eigenvalues of $\m{A}$ that are real. Having computer such a
decomposition, the calculation of the matrix exponential and matrix inverse
becomes trivial as follows:
\begin{align}
  & e^{\m{A} \dt}
  = \m{U} e^{\m{\Lambda} \dt} \transpose{\m{U}}
  = \m{U} \: \diagonal{e^{\lambda_1 \dt}}{e^{\lambda_\nn \dt}} \transpose{\m{U}} \text{ and} \elab{matrix-exponential} \\
  & \m{A}^{-1}
  = \m{U} \m{\Lambda}^{-1} \transpose{\m{U}}
  = \m{U} \: \diagonal{\lambda_1^{-1}}{\lambda_\nn^{-1}} \transpose{\m{U}}. \elab{matrix-inverse}
\end{align}

In order to obtain such an $\m{A}$, we propose to perform an auxiliary
transformation. Recall first that the conductance matrix $\m{G}$ is a symmetric
matrix, which intuitively is due to the fact that, if node $i$ is connected to
node $j$ with a certain conductance, node $j$ is also connected to node $i$ with
the same conductance (see \fref{circuit}). However, $\tm{A} = -\m{C}^{-1} \m{G}$
does not have this property. The desired symmetry can be kept intact using the
following substitution:
\begin{align} \elab{substitution}
  \begin{split}
    & \vs(t) = \m{C}^{\frac{1}{2}} \tvs(t) \text{ and} \\
    & \m{A} = -\m{C}^{-\frac{1}{2}} \m{G} \: \m{C}^{-\frac{1}{2}}
  \end{split}
\end{align}
where $\m{A}$ is symmetric since
\[
  \transpose{\m{A}}
  = -\transpose{(\m{C}^{-\frac{1}{2}} \m{G} \m{C}^{-\frac{1}{2}})}
  = -\transpose{(\m{C}^{-\frac{1}{2}})} \transpose{\m{G}} \transpose{(\m{C}^{-\frac{1}{2}})}
  = -\m{C}^{-\frac{1}{2}} \m{G} \m{C}^{-\frac{1}{2}}
  = \m{A}.
\]
Consequently, \eref{temperature-model-original} is rewritten as follows:
\begin{subnumcases}{\elab{temperature-model}}
  \frac{d\vs(t)}{dt} = \m{A} \vs(t) + \m{B} \vp(t) \elab{temperature-differential} \\
  \vq(t) = \transpose{\m{B}} \vs(t) + \vq_\ambient \elab{temperature-algebraic}
\end{subnumcases}
where
\[
  \m{B} = \m{C}^{-\frac{1}{2}} \tm{B}.
\]
Similarly, the solution in \eref{solution-original} becomes
\[
  \vs(t) = e^{\m{A} t} \vs_0 + \m{A}^{-1} (e^{\m{A} t} - \m{I}) \m{B} \v{P},
\]
and the recurrence in \eref{recurrence-original} becomes
\begin{equation} \elab{recurrence}
  \vs_i = \m{E} \vs_{i - 1} + \m{F} \vp_i
\end{equation}
for $i = \range{1}{\ns}$ where
\begin{align*}
  & \vs_0 = 0, \\
  & \m{E} = e^{\m{A} \dt}, \text{ and} \\
  & \m{F} = \m{A}^{-1} \left( e^{\m{A} \dt} - \m{I} \right) \m{B}.
\end{align*}
Using the eigendecomposition in \eref{eigendecomposition}, the last equation can
be efficiently computed in the following way:
\[
  \m{F} = \m{U} \: \diagonal{\frac{e^{\lambda_1 \dt} - 1}{\lambda_1}}{\frac{e^{\lambda_\nn \dt} - 1}{\lambda_\nn}} \transpose{\m{U}} \m{B}.
\]
As before, the recurrence in \eref{recurrence} should be followed by
\eref{temperature-algebraic} in order to obtain \mq. The performed auxiliary
transformation is helpful not only for transient analysis but also in other
contexts as we shall see later on.


\section{Static Steady-State Temperature Analysis}
\slab{static-steady-state-analysis}
The goal of static steady-state analysis is to calculate the temperature of the
thermal equilibrium that the system reaches when the power consumption stays at
a certain constant level $\mp = \vp$ for a sufficiently long time. In this case,
there are no dynamics, which means that the derivative in
\eref{temperature-differential} is zero. Then the steady-state temperature $\mq
= \vq$ can be calculated simply as
\[
  \vs = -\m{A}^{-1} \m{B} \vp,
\]
followed by \eref{temperature-algebraic} as usual, where $\m{A}^{-1}$ is
efficiently computer using \eref{matrix-inverse}.


\section{Dynamic Steady-State Temperature Analysis}
\slab{dynamic-steady-state-analysis}
Dynamic steady-state analysis addresses the shortcomings of static steady-state
analysis in one particular but important context. It tackles the scenario where
the power consumption follows a periodic pattern. In this case, after a
sufficiently long time, the system will not reach a static steady state but
instead a dynamic steady state: temperature will starts to exhibit a periodic
patter following the periodic patter of power. Then the goal of the analysis is
to find the periodic temperature profile \mq, called the dynamic steady-state
temperature profile (\up{DSSTP}), that corresponds to a given periodic power
profile \mp.

In the case of applications that exhibit periodic or close to periodic behavior,
the \up{DSSP} is of particular importance. Any design optimization has to be
performed such that the efficiency and reliability of the system at hand are
maximized considering not a relatively short transient time interval at the
system's start but the context in which the system is to function over a long
period of time, which is exactly the system's dynamic steady state.

A typical design task, for which the \up{DSSTP} is of central importance, is
temperature-aware reliability optimization. The impact of temperature on the
lifetime of electronic systems is well known \cite{srinivasan2004, coskun2006,
jedec2010, xiang2010}. The failure mechanisms commonly considered are
electromigration, time-dependent dielectric breakdown, and thermal cycling,
which are directly driven by temperature \cite{jedec2010}. What is important in
this context is that not only the average and maximum temperature have a huge
impact on the lifetime of the chip but also the amplitude and frequency of
temperature oscillations. Thus, efficient reliability optimization depends on
the availability of the actual \up{DSSTP}.

\subsection{Prior Work}
\slab{dynamic-steady-state-prior}

Let us elaborate on the techniques that have been applied in the literature in
order to compute the \up{DSSTP} as a prerequisite for reliability optimization.

A straightforward approximation of the \up{DSSTP} can be obtained by running a
temperature simulator over one or more successive periods of the application
until it is assumed that a sufficiently accurate approximation of the dynamic
steady state has been attained \cite{srinivasan2004}. The typical simulator of
choice is HotSpot \cite{skadron2003}, which is the state-of-the-art for
system-level temperature analysis \cite{srinivasan2004, liao2005, coskun2006,
liu2007, huang2009, xiang2010, thiele2011}. The simulator performs transient
temperature analysis by solving \eref{temperature-model-original} numerically
via the Runge--Kutta fourth-order method \cite{press2007}.

\inputfigure{dynamic-steady-state-prior}
The number of iterations required to reach the \up{DSSTP} depends on the thermal
characteristics of the platform. In order to illustrate this aspect, consider an
application with a period of 0.5~s running on five hypothetical platforms with
die areas of 1--25~mm\textsuperscript{2}. Let us simulate 50 successive periods
of the application via HotSpot with its default settings and compare the
resulting approximations in each period with the actual \up{DSSTP} using the
normalized root-mean-square error (\up{NRMSE}). The comparison is shown in
\fref{iterative-transient-error}. It can be observed that the number of
successive periods over which transient analysis has to be performed in order to
achieve a satisfactory level of accuracy is significant for the majority of the
configurations, which entails large computation times. For instance, for a
9-mm\textsuperscript{2} die, even after 15 iterations, the \up{NRMSE} is still
close to 20\%. Using the analytical approach to transient analysis presented in
\sref{transient-analysis}, the calculation can be sped up; however, the large
number of iterations still keeps the computation cost unreasonably high as we
shall illustrate in \sref{dynamic-steady-state-results}. Moreover, this
approximation technique provides no guarantees on the resulting accuracy since
there is no a reliable metric for measuring the proximity to the actual
solution, that is, to the actual \up{DSSTP}.

Another crude but fast approximation of the \up{DSSTP} is proposed in
\cite{huang2009}. It forgoes transient analysis all together and resides to
static steady-state analysis instead. To elaborate, it is assumed that, in each
time interval wherein the power consumption is constant, the system
instantaneously reaches a static steady state. The result of this procedure is a
stepwise temperature curve where each step corresponds to the steady-state
temperature that would be reached if the corresponding power was applied for a
sufficiently long time.

\inputfigure{static-steady-state-example}
An example of such an approximation along with the corresponding \up{DSSTP} for
an application with 10 tasks and a period of 0.1~s is given in
\fref{static-steady-state-example}. The die area is 25~mm\textsuperscript{2} in
this case. The reduced accuracy of this technique is due to the mismatch between
the actual temperature within each interval and the static steady-state
temperature. The inaccuracy depends on the thermal characteristics of the
platform and on the application itself. In order to illustrate this concern, let
us simulate five applications with periods of 0.01--1~s running on five
platforms with die areas of 1--25~mm\textsuperscript{2} and assess the resulting
profiles. The errors relative to the actual \up{DSSTP}s are shown in
\fref{static-steady-state-error}. It can be seen that, for example, for a die
area of 10~mm\textsuperscript{2} and an application period of 100~ms, the
\up{NRMSE} of this approximation technique is close to 40\%.

To conclude, the state-of-the-art techniques for dynamic steady-state
temperature analysis are slow and inaccurate. This state of affairs makes them
difficult and dangerous to be utilized for the purpose of design optimization.

\subsection{Our Solution}
\slab{dynamic-steady-state-present}

In this subsection, we are to formalize the problem of dynamic steady-state
temperature analysis and to develop an exact and, moreover, computationally
efficient solution to this problem, eliminating the issues of the
state-of-the-art solutions discussed in the previous subsection.

Consider the temperature model in \eref{temperature-model} and the corresponding
recurrence in \eref{recurrence}. The key characteristic of a dynamic steady
state is as follows:
\begin{equation} \elab{boundary-condition}
  \vs_0 = \vs_\ns.
\end{equation}
In words, the above condition means that, once the steady state has been
reached, the system starts to arrive at its initial state at the end of each
iteration, which is what makes it periodic. Then, using \eref{recurrence}, the
\up{DSSTP} can be computed by solving the following system of linear equations:
\[
  \begin{cases}
    \vs_1 - \m{E} \vs_\ns & = \m{F}_1 \vp_1 \\
    \dots \\
    \vs_\ns - \m{E}_\ns \vs_{\ns - 1} & = \m{F}_\ns \vp_\ns
  \end{cases}
\]
where the first equation enforces the boundary condition in
\eref{boundary-condition}. In order to get the big picture, the system can be
rewritten as follows:
\begin{equation} \elab{steady-state-system}
  \resizebox{0.89\linewidth}{!}{
    $\underbrace{\left[
      \begin{array}{rrrrrrr}
        \m{I}  &  0     & 0      & \cdots & 0      & 0      & -\m{E} \\
        -\m{E} & \m{I}  & 0      & \cdots & 0      & 0      & 0      \\
        0      & -\m{E} & \m{I}  & \cdots & 0      & 0      & 0      \\
        \cdots & \cdots & \cdots & \cdots & \cdots & \cdots & \cdots \\
        0      & 0      & 0      & \cdots & \m{I}  & 0      & 0      \\
        0      & 0      & 0      & \cdots & -\m{E} & \m{I}  & 0      \\
        0      & 0      & 0      & \cdots & 0      & -\m{E} & \m{I}
      \end{array}
    \right]}_{\displaystyle \mathbb{A}} \underbrace{\left[
      \begin{array}{l}
        \vs_1         \\
        \vs_2         \\
        \vs_3         \\
        \cdots        \\
        \vs_{\ns - 2} \\
        \vs_{\ns - 1} \\
        \vs_\ns
      \end{array}
    \right]}_{\displaystyle \mathbb{X}} = \underbrace{\left[
      \begin{array}{l}
        \m{F} \vp_1         \\
        \m{F} \vp_2         \\
        \m{F} \vp_3         \\
        \cdots              \\
        \m{F} \vp_{\ns - 2} \\
        \m{F} \vp_{\ns - 1} \\
        \m{F} \vp_\ns
      \end{array}
    \right]}_{\displaystyle \mathbb{B}}$
  }
\end{equation}
where $\mathbb{A}$ is an $\nn \ns \times \nn \ns$ matrix, and $\mathbb{X}$ and
$\mathbb{B}$ are $\nn \ns$-element vectors.

The most direct way to solve the system in \eref{steady-state-system} is to use
a dense solver such as the \up{LU} decomposition \cite{press2007}. However,
since $\mathbb{A}$ is a sparse matrix, a more appropriate approach is to employ
a sparse solver such as the unsymmetric multifrontal method \cite{davis2004}.
The computational complexity of such solutions is proportional to $\ns^3 \nn^3$
\cite{press2007} where $\nn$ is the number of nodes in the thermal \up{RC}
circuit, and $\ns$ is the number of samples in the power profile. The problem
here is that the system to solve can be extremely large, in particular due to
$\ns$. In such cases, direct solvers are prohibitively slow and require an
enormous amount of memory. Therefore, we do not discuss them any further.

Another potential approach is leveraging an iterative method for solving systems
of linear equations such as the Jacobi or Gauss--Seidel method \cite{press2007}.
Such methods are designed to overcome problems of direct solvers, and,
consequently, they are applicable for very large systems. However, the most
important issue with these methods is their convergence. In our experiments, we
have not observed any advantages of using these methods compared to the other
considered techniques. Thus, they are excluded from the discussion.

Yet another solution can be obtain by observing that $\mathbb{A}$ in
\eref{steady-state-system} is, in fact, a block Toeplitz matrix and, moreover, a
block-circulant matrix, wherein each block row is rotated one block element to
the right relative to the preceding block row. This observation leads to a wide
range of possible techniques for solving the system in
\eref{steady-state-system} such as the fast Fourier transform
\cite{mazancourt1983}, which we shall discuss further in our experiments in
given \sref{dynamic-steady-state-results}.

To summarize, the major problem with the aforementioned techniques is that
\one~the sparseness of $\mathbb{A}$ is not taken into account, or \two~its
specific structure is ignored, resulting in inefficient and, in some cases,
inaccurate computations. Let us now develop a solution that does not have these
issues.

Have a careful look at the structure of $\mathbb{A}$ in
\eref{steady-state-system}. The nonzero elements are located only on the block
diagonal, on the subdiagonal attached to the block diagonal, and on the
superdiagonal in the top right corner of the matrix. Linear systems with similar
structures arise in boundary value problems for ordinary differential equations,
and the typical technique to solve them is to form a so-called condensed
equation or condensed system \cite{stoer2002} as follows.

To begin with, let
\[
  \v{v}_i = \m{F} \vp_i
\]
for $i = \range{1}{\ns}$. Equation~\eref{recurrence} can then be rewritten as
follows:
\begin{equation} \elab{steady-state-recurrence}
  \vs_i = \m{E} \vs_{i - 1} + \v{v}_i
\end{equation}
for $i = \range{1}{\ns}$. Apply this formula recursively starting from $\vs_0$
leads to
\[
  \vs_i = \m{E}^i \vs_0 + \v{w}_i
\]
for $i = \range{1}{\ns}$. In the above, $\v{w}_i$ is an auxiliary recurrence
defined as
\begin{equation} \elab{steady-state-auxiliary-recurrence}
  \v{w}_i = \m{E} \v{w}_{i - 1} + \v{v}_i
\end{equation}
for $i = \range{1}{\ns}$ where
\[
  \v{w}_0 = 0.
\]
After taking \ns steps, we arrive at the following state vector:
\[
  \vs_\ns = \m{E}^\ns \vs_0 + \v{w}_\ns.
\]
Taking into account the boundary condition given in \eref{boundary-condition},
we obtain the following system of linear equations:
\[
  (\m{I} - \m{E}^\ns) \vs_\ns = \v{w}_\ns.
\]
Since $\m{E}$ is the matrix exponential, which it can be seen in
\eref{matrix-exponential}, the above system can be rewritten as follows:
\[
  (\m{I} - \m{U} e^{\m{\Lambda} \tau} \m{U}^T) \vs_\ns = \v{w}_\ns
\]
where $\tau = \ns \dt$ is the period of the power profile \mp. By splitting the
identity matrix $\m{I}$ into $\m{U} \m{U}^T$, we obtain the following solution
to the system:
\begin{align}
  \vs_\ns
  & = \m{U} (\m{I} - e^{\m{\Lambda} \tau})^{-1} \m{U}^T \v{w}_\ns \elab{t0} \\
  & = \m{U} \: \diagonal{\frac{1}{1 - e^{\lambda_1 \tau}}}{\frac{1}{1 - e^{\lambda_\ns \tau}}} \m{U}^T \v{w}_\ns. \nonumber
\end{align}
The above equation yields not only the final state vector $\vs_\ns$ but also the
initial one $\vs_0$. Consequently, the rest of the state vectors $\{ \vs_i: i =
\range{1}{\ns - 1} \}$ can be successively found by means of
\eref{steady-state-recurrence} where each $\v{v}_i$ has already been calculated
when computing $\v{w}_\ns$. The last step of the solution process is to compute
the actual temperature profile \mq, the \up{DSSTP}, by applying
\eref{temperature-algebraic}.

It can be seen that the solution to the $\nn \ns \times \nn \ns$ system given in
\eref{steady-state-system} has been reduced to the two trivial recurrences given
in \eref{steady-state-recurrence} and \eref{steady-state-auxiliary-recurrence}
that traverse the \ns steps of the power profile \mp. Therefore, the
computational complexity of this process is linear with respect to \ns, which is
important since \ns is typically much large than \nn, that is, the number of
thermal nodes.

It is worth noting that the auxiliary transformation presented in
\sref{substitution} and the accompanying eigendecomposition in
\eref{eigendecomposition} have substantially simplified the calculations
associated with dynamic steady-state analysis. It should also be noted that the
eigendecomposition along with $\m{E}$ and $\m{F}$ are computed only once for a
particular thermal \up{RC} circuit and can be considered as given together with
the circuit. In other words, these quantities stay when different power profiles
are to be analyzed, which is particularly advantageous when an intensive
design-space-exploration procedure is concerned.

\subsection{Leakage Power}

So far, we have assumed that power is independent of temperature. However, due
to the leakage component, the power dissipation is a strong function of
temperature that cannot be neglected (\sref{power-model}). Two techniques can be
applied to include in our proposed solution temperature-dependent leakage
modeling.

\subsubsection{Iterative Computation}
\slab{iterative-leakage}

In this case, we have an iterative process, depicted in \fref{leakage}, where
the temperature and power profiles are calculated in turns. With each new
temperature profile we update the power profile by computing the leakage power
and adding it to the dynamic power: $\mathbb{P}_i = \mathbb{P}_\text{dyn} +
\mathbb{P}_\text{leak}(\mathbb{T}_i)$. The process continues until the
temperature converges, i.e., the difference between two successive temperature
profiles is below a predefined bound. In our experiments we used $0.5^\circ C$
as the maximal acceptable difference and observed that the number of required
iterations to converge is 4--7.

\subsubsection{Linear Approximation}
\slab{linearized-leakage}

A linear approximation of the leakage power has the following matrix form:
$\v{P}_\text{leak}(\v{T}) = \m{A} \: \v{T}(t) + \v{B}$ where $\m{A}$ is a $N_n
\times N_n$ diagonal matrix of the proportionality and $\v{B}$ is a vector with
$N_n$ elements of the intercept. Both characterize the leakage power for each of
the $N_n$ thermal nodes in the system. It can be seen that the approximation
keeps \eref{temperature-model-original} untouched: $\m{C} \:
\frac{d\v{T}(t)}{dt} + \bar{\m{G}} \: (\v{T}(t) - \v{T}_\ambient) = \bar{\v{P}}$
where $\bar{\m{G}} = \m{G} - \m{A}$ and $\bar{\v{P}} = \v{P}_\text{dyn} + \m{A}
\: \v{T}_\ambient + \v{B}$. Therefore, all solutions proposed in this paper are
perfectly valid with the linearized model. Moreover, in spite of its simplicity,
the model provides a good estimation, as shown in \cite{liu2007}.

In order to evaluate the linearization, we have constructed a number of
hypothetical platforms with 2--32 cores (other parameters are given in
\tref{parameters}) and compared temperature profiles obtained with the
linearization and the exponential model (\sref{power-model}), respectively. For
the later, we use the iterative approach described in \sref{iterative-leakage}.
For the linearization, the power curve fitting with the least squares regression
\cite{press2007} has been employed, targeted at the range between 40 and
$80^\circ C$. From the experiments we have observed that the NRMSE is bounded by
1--2\%, indicating a good accuracy of the linear approximation.

\subsection{Experimental Results}
\slab{dynamic-steady-state-results}

In this subsection we investigate the scaling properties of the proposed
solution for the SSDTP calculation and compare it with the approach based on the
TTA with HotSpot (\sref{dynamic-steady-state-prior})\footnote{All the
experiments are performed on a Linux machine with Intel\textregistered\
Core\texttrademark\ i7-2600 3.4GHz and 8Gb of RAM.}. We also include in the
comparison two additional techniques described in the appendix, namely the TTA
with the analytical solution (\sref{transient-analysis}) and the fast Fourier
transform (FFT) (\sref{straight-forward}). In the cases of the TTA, the
simulation over successive iterations is run until the NRMSE relative to the
SSDTP obtained with the proposed method is less than 1\%.

In the following experiments, the power sampling interval is set to 1 $ms$ and
the thermal configuration of the die is the same as in \tref{parameters}. For
the experiments in this subsection, the leakage power has not been considered.
If considered according to the linearized model (\sref{linearized-leakage}),
execution times remain unchanged; if considered according to the iterative model
(\sref{iterative-leakage}), execution times increase proportionally for all the
methods, which does not affect any of the conclusions.

First, we vary the application period $\tau$ keeping the architecture fixed,
which is a quad-core platform with the core area of 4 $mm^2$. The comparison is
depicted in \fref{scaling-time} on a semilogarithmic scale. It can be seen that
the proposed technique is roughly 5000 times faster than calculating the SSDTP
by running the TTA with HotSpot and from 9 to 170 times faster than the TTA with
the analytical solution.

In the second experiment we evaluate the scaling of the proposed method with
regard to the number of processing elements. The application period is fixed to
0.5 $s$. The results are shown in \fref{scaling-cores}. It can be observed that
the proposed technique provides a significant performance improvement relative
to the alternative solutions.


\section{Power and Temperature Interdependence}
\slab{power-temperature-interdependence}

The power consumption of an electrical circuit consists of two main components:
dynamic and static. The dynamic power is due to the actual useful work done by
the circuit. The static power is due to various parasitic currents that cannot
be entirely eliminated. The major component of the static power is the leakage
power, which is especially prominent in modern \up{CMOS} transistors. The
leakage power and, therefore, the total power are strong functions of the
operating temperature, which cannot be neglected. In order to take this
phenomenon into account in the computation of the \up{DSSTP} described in
\sref{dynamic-steady-state-present}, two techniques can be applied as follows.

First of all, the power and temperature profiles can be calculated several times
in turns until they stop changing. With each new temperature profile we update
the power profile by computing the leakage power and adding it to the dynamic
power: $\mathbb{P}_i = \mathbb{P}_\text{dyn} +
\mathbb{P}_\text{leak}(\mathbb{T}_i)$. The process continues until the
temperature converges, i.e., the difference between two successive temperature
profiles is below a predefined bound. In our experiments we used $0.5^\circ C$
as the maximal acceptable difference and observed that the number of required
iterations to converge is 4--7.

A linear approximation of the leakage power has the following matrix form:
$\v{P}_\text{leak}(\v{T}) = \m{A} \: \v{T}(t) + \v{B}$ where $\m{A}$ is a $N_n
\times N_n$ diagonal matrix of the proportionality and $\v{B}$ is a vector with
$N_n$ elements of the intercept. Both characterize the leakage power for each of
the $N_n$ thermal nodes in the system. It can be seen that the approximation
keeps \eref{temperature-model-original} untouched: $\m{C} \:
\frac{d\v{T}(t)}{dt} + \bar{\m{G}} \: (\v{T}(t) - \v{T}_\ambient) = \bar{\v{P}}$
where $\bar{\m{G}} = \m{G} - \m{A}$ and $\bar{\v{P}} = \v{P}_\text{dyn} + \m{A}
\: \v{T}_\ambient + \v{B}$. Therefore, all solutions proposed in this paper are
perfectly valid with the linearized model. Moreover, in spite of its simplicity,
the model provides a good estimation, as shown in \cite{liu2007}.

In order to evaluate the linearization, we have constructed a number of
hypothetical platforms with 2--32 cores (other parameters are given in
\tref{parameters}) and compared temperature profiles obtained with the
linearization and the exponential model (\sref{power-model}), respectively. For
the later, we use the iterative approach described in
\sref{power-temperature-interdependence}. For the linearization, the power curve
fitting with the least squares regression \cite{press2007} has been employed,
targeted at the range between 40 and $80^\circ C$. From the experiments we have
observed that the NRMSE is bounded by 1--2\%, indicating a good accuracy of the
linear approximation.

\section{Reliability Optimization}
\slab{reliability-optimization}
In this section, we illustrate the importance of temperature analysis in the
context of reliability optimization. More specifically, we demonstrate the
utility of our solution to dynamic steady-state analysis presented in
\sref{dynamic-steady-state-analysis} for mitigating the fatigue due to thermal
cycling \cite{jedec2010}, which is one of the most common failure mechanism. To
this end, we develop a thermal-cycling-aware technique for scheduling of
periodic applications. (In this thesis, mapping is assumed to be a part of
scheduling. Therefore, for each task of the application in question, a schedule
prescribes not only the starting time but also the processor where the task is
supposed to be executed.)

We proceed as follows. In \sref{thermal-cycling-motivation}, the impact of
design decisions on the damage caused by thermal cycling is exemplified. The
system and reliability models that we consider are presented in
\sref{system-model} and \sref{reliability-model}, respectively. The experimental
results are reported in \sref{thermal-cycling-result}.

\subsection{Motivational Example}
\slab{thermal-cycling-motivation}

\inputfigure{thermal-cycling-task-graph}
\inputfigure{thermal-cycling-motivation}
Consider a periodic application with six tasks denoted T1--T6 and a
heterogeneous platform with two processors denoted P1 and P2. The task graph of
the application is given in \fref{thermal-cycling-task-graph} along with the
execution times for both processors. The period of the application is 60~ms. The
first alternative schedule and the corresponding \up{DSSTP} are shown on the
left-hand side of \fref{thermal-cycling-motivation} where the height of a task
represents its power consumption. It can be seen that P1 is experiencing three
thermal cycles where a thermal cycle is roughly a time interval wherein
temperature starts from a certain value, reaches an extremum, and comes back. If
we move T6 to P2, the number of cycles decreases to two, which can be seen in
the middle of \fref{thermal-cycling-motivation}. If we also swap T2 and T4, the
number of cycles of P1 drops to one, which is depicted on the right-hand side of
\fref{thermal-cycling-motivation}. According to the reliability model that we
shall describe shortly, these two changes improve the lifetime of the system by
around 45\% and 55\%, respectively, relative to the initial schedule.

The example shows that, when exploring the design space, it is important to
take into account the number of temperature fluctuations and their
characteristics. In order to obtain this information, the \up{DSSTP} has to be
computed.

\subsection{System Model}
\slab{system-model}

Consider a heterogeneous multiprocessor platform with \np processors. Processor
$i$ is characterized by a triple $(\numberof{g}_i, f_i, V_i)$ where
$\numberof{g}_i$, $f_i$, and $V_i$ are the number of gates \cite{liao2005},
frequency, and supply voltage of the processor. Consider also a periodic
application with \nt tasks. The application is given as a directed acyclic graph
$(V, E)$ where $V$ is a set of vertices corresponding to the tasks, and $E$ is a
set of edges representing data dependencies between the tasks. Let $\tau$ be the
period of the application, which is assumed to be equal to the deadline. Each
pair of a task and a processor is characterized by a tuple $(\numberof{c}_{ij},
C_{ij})$ where $\numberof{c}_{ij}$ is the number of clock cycles that task $i$
requires when it is executed on processor $j$, and $C_{ij}$ is the corresponding
effective switched capacitance.

The above information allows one to computer the power consumption of the tasks
according to the power model presented in \sref{power-model}, which will then be
used to evaluate the temperature model presented in \sref{temperature-model}.

\subsection{Reliability Model}
\slab{reliability-model}
\newcommand{\mean}{\mu}
\newcommand{\scale}{\eta}
\newcommand{\shape}{\beta}

The reliability model that we utilize is the one presented in \cite{huang2009,
xiang2010}. The model relies heavily on Weibull distributions, which we overview
in brief now.

A Weibull distribution has two parameters: $\scale$ and $\shape$, which are
called the scale and shape parameters, respectively. Let $T$ be a random
variable that is distributed according to such a distribution, which is denoted
by $T \sim \mathrm{Weibull}(\scale, \shape)$. Then the distribution function
\cite{durrett2010} of $T$ is
\begin{equation} \elab{weibull-distribution}
  F(t) = 1 - \exp\left(-\left(\frac{t}{\scale}\right)^\shape\right);
\end{equation}
the complementary distribution function of $T$ is
\begin{equation} \elab{weibull-reliability}
  R(t) = 1 - F(t) = \exp\left(-\left(\frac{t}{\scale}\right)^\shape\right);
\end{equation}
and the expectation of $T$ is
\begin{equation} \elab{weibull-expectation}
  \mean = \expectation{T} = \scale \: \Gamma\left(1 + \frac{1}{\shape}\right)
\end{equation}
where $\Gamma$ is the gamma function. In this context of reliability analysis,
$T$ represents the time to failure of the component under consideration; $F$
gives the probability of failure before a certain time moment; $R$ gives the
probability of survival up to a certain time moment, and it is called the
reliability function; and $\mean$ corresponds to the mean time to failure
(\up{MTTF}).

It is natural to expect that the distribution of $T$ is different for different
usage conditions, which is not prominent in \eref{weibull-distribution} yet. In
order to take this into account, the application period $\tau$ is split into \ns
time intervals $\{ \dt_i: i = \range{1}{\ns} \}$ so that the conditions that are
relevant to the model stay unchanged within each interval. Let $T_i \sim
\mathrm{Weibull}(\scale_i, \shape_i)$ be the time to failure that the component
would have if interval $i$ was the only interval present and denote by $\mean_i$
the corresponding \up{MTTF} according to \eref{weibull-expectation}. In the case
of temperature-induced failures, we have that $\shape_i = \shape$ for $i =
\range{1}{\ns}$, that is, all the shape parameters are equal. The reason is
that, unlike the scale parameters, the shape parameters are independent of the
operating temperature \cite{chang2006}.

As it is shown in \cite{xiang2010}, in the above scenario, the reliability
function of the component can still be approximated by means of
\eref{weibull-reliability} by letting
\[
  \scale = \frac{\sum_{i = 1}^\ns \dt_i}{\sum_{i = 1}^\ns \frac{\Delta t_i}{\scale_i}}.
\]
Applying \eref{weibull-expectation} at the level of individual cycles, the
parameter is rewritten as
\[
  \scale = \frac{\sum_{i = 1}^\ns \dt_i}{\Gamma\left(1 + \frac{1}{\shape}\right) \sum_{i = 1}^\ns \frac{\Delta t_i}{\mean_i}}.
\]
Note that the reliability model in \eref{weibull-reliability} becomes fully
specified as soon as $\dt_i$ and $\mean_i$ are identified for $i =
\range{1}{\ns}$. This part of the model depends on the particular failure
mechanism considered. In the rest of this subsection, we shall tailor the model
to the thermal-cycling fatigue, which is of interest to us due to its prominent
dependence on temperature oscillations.

In the case of thermal cycling, the time intervals with constant relevant
conditions correspond to thermal cycles. In order to detect them in a given
temperature curve, we utilize the rainflow counting method \cite{xiang2010}. As
a result, a set of \ns thermal cycles is obtained. Each detected cycle is
characterized by a number of properties including its duration, which is the
desired $\dt_i$. Regarding the corresponding $\mean_i$, it can be expressed as
follows:
\[
  \mean_i = \nc_i \dt_i
\]
where $\nc_i$ stands for the mean number of such cycles to failure. This number
is estimated using a modified version of the Coffin--Manson equation with the
Arrhenius term as shown in the following equation \cite{xiang2010, jedec2010}:
\begin{equation} \elab{reliability-mean-cycles}
  \nc = a (\Delta \q - \Delta \q_0)^{-b} \exp\left(\frac{c}{\q_\maximum}\right)
\end{equation}
where $a$, $b$, and $c$ are empirically determined constants; $\Delta \q$ is the
excursion of the cycle in question; $\Delta \q_0$ is the portion that is in the
elastic region, which does not cause damage; and $\q_\maximum$ is the maximum
temperature during the cycle.

At this point, the reliability model of the component under consideration is
fully specified. The reliability function is the one in
\eref{weibull-reliability} with
\begin{equation} \elab{reliability-scale}
  \scale = \frac{\sum_{i = 1}^\ns \dt_i}{\Gamma\left(1 + \frac{1}{\shape}\right) \sum_{i = 1}^\ns \frac{1}{\nc_i}}
\end{equation}
where $\nc_i$ is as in \eref{reliability-mean-cycles}. Using
\eref{weibull-expectation}, the \up{MTTF} of the
component is then
\begin{equation} \elab{reliability-mean-time}
  \mean = \frac{\sum_{i = 1}^\ns \dt_i}{\sum_{i = 1}^\ns \frac{1}{\nc_i}}.
\end{equation}

In conclusion, it should be noted that the reliability model requires detailed
information about the thermal cycles that the component is exposed to, which can
be obtained by performing dynamic steady-state temperature analysis.

\subsection{Problem Formulation}

The problem formulation is the following:

Given:
\begin{itemize}

\item A multiprocessor system $\Pi$ (\sref{system-model}).

\item A periodic application $G$ (\sref{system-model}).

\item The floorplan of the chip at the desired level of details, configuration
of the thermal package, and thermal parameters.

\item The parameters of the reliability model (\sref{reliability-model}), i.e.,
the constants $A$, $\Delta T_0$, $b$, $E_a$ (see
\eref{reliability-mean-cycles}).

\end{itemize}

Maximize:
\begin{equation} \elab{fitness-function}
  \mathcal{F} = \min_{i = 0}^{N_p - 1} \mean_i
\end{equation}
such that
\begin{align}
  & t_{\text{end} \: i} \leq \tau, \: \forall i \elab{deadline} \\
  & T_{ij} \leq T_\text{max}, \: \forall i, j \elab{t-max}
\end{align}
where $\mean_i$ is the MTTF of the $i$th processing element given by
\eref{reliability-mean-time}, $t_{\text{end} \: i}$ denotes the end time of
the $i$th task, $\tau$ is the period of the application, and $T_{ij}$ are
temperature values in the SSDTP. \eref{deadline} imposes the application
deadline, which we assume to be equal to the period. \eref{t-max} enforces the
constraint on the maximum temperature in the temperature profile $\mathbb{T} =
\{ T_{ij} \}$.

The optimization procedure is based on a genetic algorithm (GA)
\cite{schmitz2004} with the fitness function $\mathcal{F}$ given by
\eref{fitness-function}. The algorithm is outlined in \sref{genetic-algorithm}.

This section contains the derivation of the reliability model discussed in
\sref{reliability-optimization} and the description of the actual optimization
procedure.

\subsection{Optimization Procedure}
\slab{genetic-algorithm}

The optimization procedure is based on a genetic algorithm \cite{schmitz2004}
with the fitness function $\mathcal{F}$ given by \eref{fitness-function}. Each
chromosome is a vector of $2 \times N_t$ elements, where the first half encodes
priorities of the tasks and the second represents a mapping. The population
contains $4 \times N_t$ individuals that are initialized partially randomly and
partially based on the initial temperature-aware solution \cite{xie2006}. In
each generation, a number of individuals, called parents, are chosen for
breeding by the tournament selection with the number of competitors proportional
to the population size. The parents undergo the 2-point crossover with $0.8$
probability and uniform mutation with $0.01$ probability. The evolution
mechanism follows the elitism model where the best individual always survives.
The stopping condition is an absence of improvement within 200 successive
generations.

The fitness of a chromosome, \eref{fitness-function}, is evaluated in a number
of steps. First, the decoded priorities and mapping are given to a list
scheduler that produces schedules for each of the cores. If the application
schedule does not satisfy the deadline, the solution is penalized proportionally
to the delay and is not further evaluated; otherwise, based on the parameters of
the architecture and tasks, a power profile is obtained and the corresponding
SSDTP is computed by our proposed method. If the SSDTP violates the temperature
constraint given by \eref{t-max}, the solution is penalized proportionally to
the amount of violation and not further processed; otherwise, the MTTF of each
core is estimated according to \eref{reliability-mean-time} and the fitness
function $\mathcal{F}$ is computed.

\subsection{Experimental Results}
\slab{thermal-cycling-result}

In this section we evaluate the reliability optimization approach described in
\sref{reliability-optimization}, first with a set of synthetic applications and,
finally, using a real-life example.

The experimental setup is the following. Heterogeneous platforms and periodic
applications are generated randomly \cite{dick1998} in such a way that the
execution time of tasks is uniformly distributed between 1 and 10 $ms$ and the
leakage power accounts for 30--60\% of the total power dissipation\footnote{The
parameters of the applications and platforms (task graphs, floorplans, HotSpot
configurations, etc.) used in our experiments are available online at
\cite{liu2011}.}. The linear leakage model is used in the experiments, since, as
discussed in \sref{power-temperature-interdependence}, it provides a good
approximation. The area of one core is 4 $mm^2$, other parameters of the die and
thermal package are given in \tref{parameters}. The temperature constraint
$T_\text{max}$ (see \eref{t-max}) is set to $100^\circ C$. In
\eref{reliability-mean-cycles} the Coffin-Manson exponent $b$ is set to 6, the
activation energy $E_a$ to 0.5, and the elastic temperature region $\Delta T_0$
to zero \cite{jedec2010}. The coefficient of proportionality $A$ is not
significant, since we are concerned about the relative improvement.

In each of the experiments, we compare the optimized solution with an initial
temperature-aware solution proposed in \cite{xie2006}. This solution consists of
a task mapping and schedule that captures the spatial temperature behavior and
tries to minimize the peak temperature while satisfying the real-time
constraints. The deadline is set to the duration of the initial schedule
extended by 5\%.

In the first set of experiments, we change the number of cores $N_p$ while
keeping the number of tasks $N_t$ per core constant and equal to 20. For each
problem we have generated 20 random task graphs and found the average
improvement of the MTTF over the initial solution ($\scriptstyle
\text{MTTF}_\times$). We also have measured the change in the consumed energy
($\scriptstyle \text{E}_\times$). The results are given in \tref{mttf-cores}
($t$ indicates the optimization time in seconds). It can be seen that the
reliability-aware optimization dramatically increases the MTTF by 13 up to 40
times. Even for large applications with, e.g., 320 tasks deployed onto 16 cores,
a feasible mapping and schedule that significantly improve the lifetime of the
system can be found in an affordable time. Moreover, our optimization does not
impact the energy efficiency of the system.

For the second set of experiments, we keep the quad-core architecture and vary
the size (number of tasks $N_t$) of the application. The number of randomly
generated task graphs per application size is 20. The average improvement of the
MTTF along with the change in the energy consumption are given in
\tref{mttf-tasks}. The observations are similar to those for the previous set of
experiments.

The above experiments have confirmed that our proposed approach is able to
effectively increase the MTTF of the system. The efficiency of this approach is
due to the fast and accurate SSDTP calculation, which is at the heart of the
optimization, and which, due to its speed, allows a huge portion of the design
space to be explored. In order to prove this, we have replaced, inside our
optimization framework, the proposed SSDTP calculation with the calculation
based on HotSpot and based on the SSA, respectively
(\sref{dynamic-steady-state-prior}). The goal is to compare our results with the
results produced using HotSpot and the SSA, after the same optimization time as
needed with the proposed SSDTP calculation technique. The experimental setup is
the same as for the experiments in \tref{mttf-tasks}. The MTTF obtained with
HotSpot and the SSA is evaluated and compared with the MTTF obtained by our
proposed method. The results are summarized in \tref{mttf-comparison}. For
example, the lifetime of the platform running 160 tasks can be extended by more
than 18 times, compared to the initial solution, using our approach, whereas,
the best solutions found with HotSpot and the SSA, using the same optimization
time, are only 2.02 and 5.33 times better, respectively. The reason for the poor
results with HotSpot is the excessively long execution time of the SSDTP
calculation. This allows for a much less thorough investigation of the solution
space than with our proposed technique. In the case of the SSA, the reason is
different. The SSA is fast but also very inaccurate
(\sref{dynamic-steady-state-prior}). The inaccuracy drives the optimization
towards solutions that turn out to be of low quality.

We have seen that our reliability-targeted optimizations have significantly
increased the MTTF without affecting the energy consumption. This is not
surprising, since our optimization will search towards low temperature
solutions, which implicitly means low leakage. In order to further explore this
aspect, we have performed a multi-objective optimization\footnote{The
multi-objective optimization is based on NSGA-II \cite{deb2002}.} along the
dimensions of energy and reliability. An example of the Pareto front averaged
over 20 applications with 80 tasks deployed onto a quad-core platform is given
in \fref{average-pareto}. It can be observed that the variation of energy
is less than 2\%. This means that solutions optimized for the MTTF have an
energy consumption almost identical to those optimized for energy. At the same
time, the difference along the MTTF is huge. This means that ignoring the
reliability aspect one may end up with a significantly decreased MTTF, without
any significant gain in energy.

Finally, we have applied our optimization technique to a real-life example,
namely the MPEG2 video decoder \cite{ffmpeg2011} that is deployed onto a
dual-core platform. The decoder was analyzed and split into 34 tasks. The
parameters of each task were obtained through a system-level simulation using
MPARM \cite{benini2005}. The deadline is set to 40 $ms$ assuming 25 video frames
per second. The solution found with the proposed method improves the lifetime of
the system by 23.59 times with a 5\% energy saving, compared to the initial
solution. The same optimization was solved using HotSpot and the SSA. The best
found solutions are only 5.37 and 11.50 times better than the initial one,
respectively.

Experiments demonstrate the superiority of the proposed techniques, compared to
the state of the art.



\section{Conclusion}

In this paper we have proposed an efficient and accurate technique to calculate
the SSDTP of an embedded multiprocessor system. Using the proposed approach, we
conducted a temperature-aware reliability optimization based on the thermal
cycling failure mechanism and have shown that taking into consideration the
temperature variations within a multicore platform can significantly prolong its
lifetime without affecting its energy efficiency. The improvement, compared
using the state of the art, is significant.
