In this paper, we have presented a framework for probabilistic analysis of
electronic systems. Given a description of the probability distribution of the
uncertain parameters present in the system under consideration and a simulator
of a metric of interest dependent on the parameters, the framework prescribes
the steps that need to be taken in order to computationally efficiently obtain
the probability distribution of the metric.

The proposed approach is powered by hierarchical interpolation following a
hybrid adaptation strategy. The adaptivity makes the framework particularly
suited for problems with idiosyncratic behaviors and steep response surfaces,
which arise in electronic systems due to their digital nature.

The performance of our framework has been assessed by comparing it with the
performance of an advanced sampling technique. The experimental results have
shown that, for a fixed budget of evaluations of the metric, our approach
achieves higher accuracy compared to direct simulations.

Finally, we would like to emphasize that, even though the framework has been
exemplified by considering a specific source of uncertainty and specific
metrics, it is general and can be successfully applied in many other settings.

We develop a framework for system-level analysis of electronic systems whose
runtime behaviors depend on uncertain parameters. The proposed approach thrives
on hierarchical interpolation guided by an advanced adaptation strategy, which
makes the framework general and suitable for studying various metrics that are
of interest to the designer. Examples of such metrics include the end-to-end
delay, total energy consumption, and maximum temperature of the system under
consideration. The framework delivers a light generative representation that
allows for a straightforward, computationally efficient calculation of the
probability distribution and accompanying statistics of the metric at hand. Our
technique is illustrated by considering a number of uncertainty-quantification
problems and comparing the corresponding results with exhaustive simulations.
