Consider an electronic system composed of two major components: a platform and
an application. The platform is a collection of heterogeneous processing
elements as defined in \sref{system-model}, whereas the application is a
collection of interdependent tasks. The designer is interested in studying a
quantity \g that characterizes the system at hand from a certain perspective.
Examples of \g include the execution delay of the application or of a specific
task and the total energy consumption of the platform or of a specific
processing element.

The quantity of interest \g depends on a set of parameters \vu that are
uncertain at the design stage. Examples of \vu include the amount of data that
the application must process, execution times of the tasks, and properties of
the environment. The parameters \vu are given as a random vector $\vu: \Omega
\to \real^\nu$, which is defined on $(\Omega, \F, \probability)$ as in
\xref{probability-theory}, with an arbitrary but known distribution, whose
\ac{CDF} is denoted by $F$.

The dependency of \g on \vu implies that \g is random to the designer. For a
given outcome of \vu, however, the evaluation of \g is assumed to be purely
deterministic. This operation is traditionally undertaken by an adequate system
simulator, and it is considered doable but computationally expensive.

Our objective in this chapter is to develop a framework for calculating the
probability distribution of the quantity of interest \g dependent on the
uncertain parameters \vu so that this framework is able to efficiently handle
nondifferentiable and potentially discontinuous dependencies between \g and \vu,
which constitutes an important class of problems for electronic-system design.
