Similarly to \cref{uncertainty-process-development}, we propose a design-time
system-level framework for analysis of electronic systems that depend on
uncertain parameters. In this chapter, however, the source of uncertainty being
considered is workload. As is the case with sampling methods (see \sref{past}),
our technique treats the system at hand as a ``black box'' and thus is
straightforward to apply in practice, since no handcrafting is required, and
existing code need not be changed. Hence, the quantities that the framework is
able to address are diverse, including those that are concerned with timing-,
power-, and temperature-related characteristics of applications running on
heterogeneous platforms.

In contrast to \cref{uncertainty-process-development} and sampling methods, the
framework presented in this chapter explores and exploits the nature of the
problem---that is, the way the quantity of interest depends upon the uncertain
parameters---by exercising the aforementioned ``black box'' at an adaptively
chosen set of points. The adaptivity that we leverage is hybrid
\cite{jakeman2012}: it is sensitive to both global (on the level of individual
stochastic dimensions \cite{klimke2006}) and, more importantly, local (on the
level of individual points \cite{ma2009}) variations. This means that the
framework is able to benefit from any peculiarities that might be present in the
stochastic space, the space of the uncertain parameters. The adaptivity is the
primary feature of our technique, which we discuss and illustrate next.
