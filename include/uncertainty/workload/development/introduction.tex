Similarly to \cref{uncertainty-process-development}, we propose a design-time
system-level framework for the analysis of electronic systems that depend on
uncertain parameters. As it is the case with sampling methods introduced in
\sref{past}, our technique treats the system at hand as a ``black box'' and thus
is straightforward to apply in practice, since no handcrafting is required, and
existing codes need no change. Consequently, the quantities that the framework
is able to address are diverse, including those that are concerned with timing-,
power-, and temperature-related characteristics of applications running on
heterogeneous platforms.

In contrast to \cref{uncertainty-process-development} and sampling methods, the
framework presented in this chapter explores and exploits the nature of the
problem---that is, the way the quantity of interest depends on the uncertain
parameters---by exercising the aforementioned ``black box'' at a set of points
chosen adaptively. The adaptivity that we leverage is hybrid \cite{jakeman2012}:
it is sensitive to both global (on the level of individual stochastic dimensions
\cite{klimke2006}) and, more importantly, local (on the level on individual
points \cite{ma2009}) variations. This means that the framework is able to
benefit from any particularities that might be present in the stochastic space,
the space of the uncertain parameters. The adaptivity is the capital feature of
our technique, which we motivate and illustrate next.
