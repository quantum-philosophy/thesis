Process variation is an acute concern of computer-system designs
\cite{chandrakasan2000, srivastava2010}. A crucial implication of process
variation is that it renders the key parameters of a technological
process---such as the effective channel length, gate oxide thickness, and
threshold voltage---as uncertain quantities. Therefore, the same workload
applied to two seemingly identical dies can lead to two drastically different
power profiles and, thus, to two drastically different temperature profiles
since power consumption and heat dissipation depend on the aforementioned
quantities. This concern is especially exigent due to the interplay between the
static power and temperature \cite{liu2007, srivastava2010} discussed in
\sref{power-model}. As it is the case with arguably any uncertainty in computer
systems, the one due to process variation can lead to deterioration in
efficiency and to faults of various magnitudes, and, consequently, process
variation should be adequately analyzed as the foremost step toward efficient
and robust products.

An important problem in this regard is the characterization of the on-wafer
distribution of a quantity of interest that is deteriorated by process
variation, based on measurements. The problem belongs to the class of inverse
problems since the measured data can be seen as an output of the system at hand,
and the desired quantity as an input. Such an inverse problem is addressed here.

Our goal is to characterize arbitrary process parameters with high accuracy and
at low costs. The goal is accomplished by measuring auxiliary quantities that
are more convenient and less expensive to work with and then employing
statistics in order to infer the desired parameters from the measurements. More
specifically, we propose a novel approach to the quantification of process
variation based on indirect, incomplete, and noisy measurements. Moreover, we
develop and implement a solid framework around the proposed idea and perform a
thorough study of various aspects of our technique.

The remainder of the chapter is structured as follows. A motivational example is
given in \sref{inference-motivation}. In \sref{inference-problem}, we formulate
the problem that we address as well as the requirements to a potential solution.
The relevant prior studies are discussed in \sref{inference-prior}. The solution
that we propose is presented in \sref{inference-solution}, and the corresponding
experimental results are reported and elaborated on in \sref{inference-results}.
\sref{inference-conclusion} concludes this chapter.
