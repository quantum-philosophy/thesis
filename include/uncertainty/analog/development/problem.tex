Assume the system model given in \sref{system-model}. Suppose that the system
depends on a number of process parameters that are uncertain at the design
stage. Once the fabrication process yields a particular outcome, the considered
process parameters take certain values and stay unchanged thereafter. However,
these certain values are different for different fabricated chips, and they also
vary within each fabricated chip. In particular, the variability leads to
deviations of power from the nominal values and, therefore, to deviations of
temperature from the one corresponding to the nominal power consumption.

Each process parameter is a characteristic of a single transistor; consider, for
instance, the effective channel length. Therefore, each device in the electrical
circuit at hand can potentially have a different value of this parameter since,
in general, the variability due to process variation is not uniform.
Consequently, the parameters can be modeled as a stochastic process
\[
  \u: \Omega \times \real^2 \to \real^\nu
\]
defined on a suitable probability space $(\Omega, \F, \probability)$ (see
\xref{probability-theory}) and a two-dimensional plane and taking values in
$\real^\nu$ where \nu is the number of the considered process parameters. Since
our work is system-level oriented, the stochastic process is discretized, and
each processing element is modeled with a finite set of random variables. Then
the uncertain parameters of the problem under consideration are defined as
\[
  \vu = (\u_i)_{i = 1}^\nu: \Omega \to \real^\nu
\]
so that the individual random variables of the processing elements are collected
into a single random vector where \nu is redefined to be the total number of
these variables. Given this setting, our goal in this chapter is twofold.

First, we are to develop a system-level framework for transient temperature
(and, hence, power) analysis as well as dynamic steady-state analysis of
electronic systems where power consumption and heat dissipation are stochastic
due to their dependency on the parameters \vu. The designer is required to
specify \one~the probability distribution of \vu and \two~the dependency of the
system's power consumption on \vu, which can be given as a ``black box.'' The
framework is to provide the designer with tools for analyzing the system under a
given workload---without imposing constraints on the nature of this
workload---and calculating the corresponding stochastic power \mp and stochastic
temperature \mq profiles with a desired level of accuracy and at low
computational costs.

Second, taking into consideration the effect of process variation on power and
temperature, we are to find the reliability function of the system and to
develop a computationally efficient design-space-exploration scheme exploiting
the proposed techniques for power, temperature, and reliability analysis.
