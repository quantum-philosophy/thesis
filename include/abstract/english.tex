One major concern of the designer of electronic systems is the presence of
uncertainty, which is due to such phenomena as process and workload variation.
In many cases, uncertainty is inherent and inevitable, and it can lead to
degradation of the quality of service in the best case and to severe faults or
burnt silicon in the worst-case scenario. Therefore, it is crucial to analyze
uncertainty and to mitigate its damaging consequences by designing electronic
systems in such a way that they effectively and efficiently take uncertainty
into account.

We begin by considering techniques for deterministic system-level analysis and
design of certain aspects of electronic systems. These techniques do not take
uncertainty into account; however, they serve as a solid foundation for those
that do. Our attention revolves primarily around power and temperature as they
are of central importance for attaining robustness and energy efficiency. We
develop a novel approach to dynamic steady-state temperature analysis of
electronic systems and apply it in the context of reliability optimization.

We then proceed to developing techniques that address uncertainty. The first
technique of this kind is to quantify the variability of process parameters,
which is induced by process variation, across silicon wafers based on indirect
and potentially incomplete and noisy measurements. The second technique is to
study diverse system-level characteristics with respect to the variability
originated from process variation. In particular, it allows for analyzing
transient power and temperature profiles as well as dynamic steady-state power
and temperature profiles of electronic systems. This is illustrated by
considering a problem of design-space exploration with probabilistic constraints
related to reliability. The third technique that we develop is to efficiently
tackle the case of less regular sources of uncertainty than process variation
such as workload variation. This technique is exemplified by analyzing the
effect that workload units with uncertain processing times have on the timing-,
power-, and temperature-related characteristics of the system under
consideration.

We address also the issue of runtime management of electronic systems under
uncertainty. In this context, we perform an early investigation of the utility
of advanced prediction techniques for the purpose of fine-grained long-range
forecasting of the resource usage in large electronic systems.

All the proposed techniques are assessed by extensive experimental evaluations,
which demonstrate the superior performance of our approaches to analysis and
design of electronic systems with respect to existing techniques.

\vspace{1em}
\noindent
\emph{
  The research presented in this thesis has been partially funded by the
  National Computer Science Graduate School (\up{CUGS}) in Sweden.
}
