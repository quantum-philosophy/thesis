Ett stort problem för designern inom elektroniska system är förekomsten av
osäkerhet, som beror på sådana fenomen som variationer relaterade till
tillverkning och arbetsbelastning. Osäkerhet är i många fall naturlig och
oundviklig och kan leda till försämring av servicekvaliteten i bästa fall och
till allvarliga fel eller bränd kisel i värsta fall. Därför är det viktigt att
analysera osäkerhet och att mildra dess skadliga följder genom att designa
elektroniska system på sådant sätt att de effektivt och ändamålsenligt tar
hänsyn till osäkerhet.

Vi börjar med att överväga tekniker för deterministisk systemnivåanalys och
systemnivådesign av elektroniska system. Dessa tekniker tar inte hänsyn till
osäkerhet; de fungerar dock som en solid grund för de tekniker som gör det. Vi
fokuserar främst på faktorer som kraft och temperatur eftersom de är av central
betydelse för att uppnå robusthet och energieffektivitet. Vi utvecklar ett nytt
tillvägagångssätt för dynamisk stabiliserad temperaturanalys av elektroniska
system och tillämpar det inom ramen för tillförlitlighetsoptimering.

Vi fortsätter sedan med att utveckla ett antal tekniker som tar hänsyn till
osäkerhet i elektroniska system. Den första tekniken är utformad för att
kvantifiera föränderligheten hos processparametrar, som framkallas av
processvariation, över kiselplattor baserat på indirekta och potentiellt
ofullständiga och bullriga mätningar. Den andra tekniken går ut på att studera
olika systemnivåegenskaper med avseende på variabiliteten som härrör från
processvariation. I synnerhet tillåter den analys av övergående
temperaturprofiler samt dynamiska stabiliserade temperaturprofiler hos
elektroniska system. Detta illustreras genom att överväga problemet med
utforskning av ett designutrymme med probabilistiska begränsningar relaterade
till tillförlitlighet. Den tredje tekniken som vi utvecklar är utformad för att
effektivt ta itu med osäkerhetsfaktorer som är mindre regelbundna än
processvariation, som till exempel variationer i arbetsbelastning. Denna teknik
exemplifieras genom att analysera den effekt som arbetsbelastningsenheter med
osäkra behandlingstider har på det aktuella systemets tids-, kraft- och
temperaturrelaterade egenskaper.

Vi tar även hänsyn till frågan om körtidshantering av elektroniska system under
osäkerhet. I det här sammanhanget utför vi en tidig undersökning av
användbarheten av avancerade prediktiva tekniker för att anskaffa en förfinad
prognos för långsiktig användning av resurser i stora datorsystem.

Alla föreslagna tekniker bedöms genom omfattande experimentella utvärderingar
som påvisar den överlägsna prestationsförmågan av våra metoder för analys och
design av elektroniska system med avseende på befintliga tekniker.

% vim: set spelllang=sv:
