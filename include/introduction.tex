Computer systems are omnipresent and omniscient. They penetrate deep into
everyday life and are unsettlingly indispensable and increasingly intelligent at
the tasks entrusted to them. It is then understandable that analysis and design
of computer systems are acutely difficult and vastly far-reaching endeavors.

One major concern of the designer of a computer system is the presence of
uncertainty, which, in many cases, is inherent and inevitable. Uncertainty can
be due to different phenomena; it can originate from different sources. From one
of many perspectives, uncertainty in computer systems can be broadly classified
into two categories: analog and digital, which---in order to build the reader's
intuition---can also be referred to as physical and virtual, respectively.

A prominent example of an analog source of uncertainty is process variation
\cite{srivastava2010}. In this case, uncertainty originates from the
imperfection of the contemporary fabrication process. Namely, the process
parameters of fabricated nanoscale devices deviate from their nominal values as
the fabrication process cannot be controlled precisely down to the level of
individual atoms. Process variation is a topic of many lines of research; see,
for instance, \cite{bhardwaj2006, bhardwaj2008, chandra2010, juan2012, lee2013}.

Another example of an analog source of uncertainty that we would like to mention
is what we refer to as performance variation. In this case, the uncertainty is
due to the performance of electrical circuits degrading over time due to natural
or accelerated wear \cite{jedec2016}, which typically ends in a permanent fault
and, hence, limits the lifetime of the system at hand. The degradation is
nonuniform, and the lifetime is unknown in advance. Performance uncertainty has
been extensively studied as well; see, for instance, \cite{coskun2006,
huang2009b, das2014c}.

A prominent example of a digital source of uncertainty is workload. In this
case, the uncertainty is due to the varying characteristics of codes running on
computer systems. To elaborate, from one activation to another, the same piece
of deterministic software can exhibit drastically different behaviors depending
on the environment and input data, and neither the environment nor input data
that the system under development will be exposed to is exhaustively known in
advance. Workload uncertainty is arguably the most consequential type of
uncertainty and, thus, has not been deprived of attention, especially in the
real-time community; see, for instance, \cite{diaz2002, santinelli2011,
quinton2012, tanasa2015}.

Such phenomena as the ones mentioned above render the behavior of a computer
system nondeterministic to the designer of this system. The presence of
uncertainty leads to degradation of the quality and efficiency of service in the
best case and to severe faults or burnt silicon in the worst-case scenario.
Consequently, it is crucial to acknowledge and analyze uncertainty in computer
systems and to mitigate its deteriorating consequences by designing these
systems in such a way that they are well aware of uncertainty and well equipped
with mechanisms to effectively and efficiently tackle it.

\section{Overview}
\inputsection{overview}

\section{Publications}
\inputsection{publications}
