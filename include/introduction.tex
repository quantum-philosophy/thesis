In this first chapter, we provide the motivation and overview of the thesis.

\section{Motivation}

Computer systems are omnipresent and omniscient. They penetrate deep into
everyday life and are unsettlingly indispensable and increasingly intelligent at
tasks entrusted to them. It is readily understandable that analysis and design
of computer systems are acutely difficult and vastly far-reaching endeavors.

One major concern of the designer of computer systems is the presence of
uncertainty, which, in many cases, is inherent and inevitable. Uncertainty can
be due to different phenomena; it can originate from different sources. From one
of many perspectives, uncertainty in computer systems can be broadly classified
into analog and digital. Analog uncertainty originates from the physical world
while digital uncertainty originates from the virtual, computer world.

A prominent example of analog uncertainty is the one due to process variation
\cite{srivastava2010}. In this case, the source of uncertainty is the
imperfection of the contemporary fabrication process: the process parameters of
fabricated nanoscale devices deviate from their nominal values since the
fabrication process cannot be controlled precisely down to the level of
individual atoms. Process variation is a topic of many lines of research; see,
for instance, \cite{bhardwaj2006, bhardwaj2008, chandra2010, juan2012, lee2013}.

Another example of analog uncertainty is the one due to what we refer to as
performance variation. In this case, uncertainty arises from natural or
accelerated wear or fatigue \cite{jedec2016}. The performance of an electrical
circuit degrades over time, and this physical deterioration typically ends in a
permanent fault and, therefore, limits the lifetime of the system at hand. The
degradation is nonuniform, and the lifetime is unknown in advance. Performance
uncertainty has been extensively studied as well; see, for instance,
\cite{coskun2006, huang2009b, das2014c}.

A salient example of digital uncertainty is the one due to workload variation.
In this case, the source of uncertainty is the actual work that computer systems
are required to perform. From one activation to another, the same piece of
deterministic software can exhibit drastically different behaviors depending on
the environment and input data, and neither the environment nor input data that
the system under development will be exposed to at runtime is exhaustively known
in advance. Workload uncertainty is a highly consequential type of uncertainty
and, therefore, has not been deprived of due attention, especially in the
real-time community; see, for instance, \cite{diaz2002, santinelli2011,
quinton2012, tanasa2015}.

Such phenomena as the ones mentioned above render the behavior of computer
systems nondeterministic to the designer of these systems. The presence of
uncertainty leads to degradation of the quality and efficiency of service in the
best case and to severe faults or burnt silicon in the worst-case scenario.
Consequently, it is crucial to acknowledge and analyze uncertainty in computer
systems and to mitigate its deteriorating consequences by designing these
systems in such a way that they are well aware of uncertainty and well equipped
with mechanisms to effectively and efficiently take it into consideration.

\section{Overview}
\inputsection{overview}

\section{Publications}
\inputsection{publications}
