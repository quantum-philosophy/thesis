The goal of the thesis is to assist the designer of computer systems by
providing effective and efficient tools for quantifying and mitigating
uncertainty. To this end, we develop a number of system-level techniques for
analyzing and designing under process, performance, and workload uncertainty
introduced in \sref{motivation}. Specifically, our work makes the following
contribution.

\begin{itemize}

\item
We improve deterministic transient power and temperature analysis and propose an
accurate and fast approach to deterministic dynamic steady-state analysis.
Leveraging our efficient approach, we develop a reliability-optimization scheme
targeted at reducing the stress due to thermal cycling \cite{jedec2016} and,
thereby, mitigating performance uncertainty.

\item
We present a versatile statistical framework for inferring the variability of
process parameters across silicon wafers that is induced by process variation.
The framework operates on indirect measurements---such as readings from thermal
sensors---which can also be sparse and noisy.

\item
We develop a flexible probabilistic framework for characterizing diverse
system-level quantities of interest that are deteriorated by process variation.
Using our technique, we enhance reliability analysis of computer systems so that
the damaging ramifications of process uncertainty are adequately accounted for
in reliability models, which also leads to a more comprehensive treatment of
performance uncertainty.

\item
We introduce an adept system-level framework for probabilistic analysis of
various quantities that are of interest but also uncertain to the designer due
to the variability induced by digital sources of uncertainty such as the runtime
workload, which tends to be less regular than the one originating from analog
sources of uncertainty such as the fabrication process.

\item
We perform an early but insightful investigation of the applicability of the
latest advancements in the field of machine learning to mitigating workload
uncertainty at runtime in the context of resource management and, more
specifically, to the problem of fine-grained long-range forecasting of the
resource usage in large computer systems.

\end{itemize}
