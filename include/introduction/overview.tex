A bird's eye view of the content of the thesis is as follows.

In \cref{certainty-development}, we consider techniques for deterministic
system-level analysis of computer systems. These techniques do not take
uncertainty into account; however, they serve as a solid foundation for those
that do. Our attention revolves primarily around power and temperature since
they are of central importance for attaining robustness and energy efficiency.
We develop an accurate and fast approach to dynamic steady-state temperature
analysis of multiprocessor systems and apply it in the context of
temperature-aware reliability optimization. Reliability optimization addresses
performance uncertainty by definition; however, it falls in the scope of
\cref{certainty-development} since the accompanying temperature-aware
reliability analysis treats temperature as a deterministic quantity, which is
suboptimal as we discuss in the subsequent chapters.

In \cref{uncertainty-analog-fabrication}, we present our first technique that
addresses uncertainty. Specifically, we develop an approach to analyzing the
variability of process parameters that is induced by process variation, across
silicon wafers by means of indirect and sparse measurements such as readings
from thermal sensors.

In \cref{uncertainty-analog-development}, we continue working with process
uncertainty and present a technique that is applicable to studying diverse
quantities with respect to process variation. In particular, the approach allows
for progressively analyzing transient power and temperature profiles of
multiprocessor systems. Other examples are the dynamic steady-state temperature
profile and such critical metrics as the maximal temperature and energy
consumption. They all can be analyzed from the probabilistic standpoint, and the
utility of this virtue is illustrated by addressing a problem of design-space
exploration with probabilistic constraints related to reliability. Unlike the
reliability model given in \cref{certainty-development}, the one presented in
\cref{uncertainty-analog-development} is well aware of process uncertainty and,
hence, allows for a more adequate treatment of performance uncertainty.

In \cref{uncertainty-digital-development}, we develop another system-level
technique; this one, however, is efficient at tackling digital sources of
uncertainty such as workload variation, which tend to be less regular than the
analog ones including process variation. The proposed technique is exemplified
by quantifying the effect that workload units with uncertain processing times
have on the timing-, power-, and temperature-related characteristics of
multiprocessor systems.

In \cref{uncertainty-digital-management}, we contemplate the future of runtime
management of computer systems under workload variation. In this context, we
perform an early investigation of the utility of advanced prediction techniques
for fine-grained long-range forecasting of the resource usage in large computer
system.

In \cref{conclusion}, we conclude the thesis with a few final remarks.
