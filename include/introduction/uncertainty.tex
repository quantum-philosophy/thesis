Computer systems are subject to uncertainty. Uncertainty can be due to different
phenomena, and, in many cases, it is inherent and inevitable. From one of many
perspectives, uncertainty in computer systems can be broadly classified into two
categories: analog and digital. Analog uncertainty emerges from the physical
world while digital uncertainty emerges from the virtual, computer world. In
order clarify this classification, let us consider three examples.

\subsection{Process Variation}
\slab{process-variation}

A prominent example of analog uncertainty is the one stemming from process
variation \cite{chandrakasan2000, srivastava2010}. In this case, the source of
uncertainty is the fabrication process. Specifically, the process parameters of
fabricated nanoscale devices deviate from their nominal values since the
contemporary fabrication process cannot be controlled precisely down to the
level of individual atoms.

The aforementioned transistor-level variability propagates to such crucial
system-level characteristics of a computer system as power consumption and heat
dissipation and, thereby, makes them uncertain to the designer. The propagation
is due to process variation affecting the key parameters of a technological
process such as the effective channel length, gate oxide thickness, and
threshold voltage. As a result, the same workload applied to two seemingly
identical dies can lead to two drastically different power profiles and,
consequently, to two drastically different temperature profiles since power
consumption and heat dissipation depend on the aforementioned quantities. This
concern is especially exigent due to the interdependence between power and
temperature---which is to be discussed further in \sref{power-model} as well as
\sref{power-temperature-interplay}---whose magnitude depends on process
parameters.

\subsection{Performance Variation}
\slab{performance-variation}

Another example of analog uncertainty is the one originating from performance
variation. In this case, uncertainty arises from natural or accelerated wear or
fatigue \cite{jedec2016}, that is, from the performance of electrical circuits
degrading over time. This physical degradation can cause a fatal fault and,
hence, abruptly end the life of the system at hand. Since the degradation is a
nonuniform and intricate process, the system's lifetime is uncertain to the
designer.

\subsection{Workload Variation}
\slab{workload-variation}

A salient example of digital uncertainty is the one emerging from workload
variation. In this case, the source of uncertainty is the actual work that
computer systems are instructed to perform. To elaborate, from one activation to
another, the same piece of deterministic software can exhibit drastically
different behaviors depending on the environment and input data, and neither the
environment nor input data that the system under consideration will be exposed
to at runtime is exhaustively known at early development stages.

\conclusioncut
In conclusion of this section, let us draw attention to the following aspect.
Process variation has been a topic of many lines of research; see, for instance,
\cite{bhardwaj2006, bhardwaj2008, chandra2010, juan2012, lee2013}. Similarly,
performance uncertainty has extensively been studied in the literature; see, for
instance, \cite{coskun2006, huang2009b, das2014c}. Workload uncertainty has not
been deprived of due attention either, especially in the real-time community;
see, for instance, \cite{diaz2002, santinelli2011, quinton2012, tanasa2015}.
However, as we discuss in detail in the relevant parts of the thesis, certain
problems have not been addressed yet, and the solutions to those that have are
strictly restricted in use, which is frequently due the unrealistic assumptions
that these solutions make.

Having introduced and exemplified uncertainty in computer systems, we are ready
to consolidate our motivation and solidify our objective.
