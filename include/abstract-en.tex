Computer systems are omnipresent and omniscient. They penetrate deep into
everyday life and are unsettlingly indispensable and increasingly intelligent at
the tasks delegated to them. It is readily understandable that computer-system
design is a difficult process, and that this process is vastly consequential.

One of the major concerns that the designer of a computer system has to address
is the presence of uncertainty, which, in many cases, is inherent and
inevitable. One prominent source of uncertainty is process variation: the
physical properties of each fabricated device deviate from the corresponding
specification since the contemporary fabrication process cannot be controlled
precisely down to individual atoms. Another source is performance variation: the
performance of the system under consideration degrades over time due to natural
or accelerated wear. Yet another and arguably the most consequential source of
uncertainty is workload variation: the actual runtime load of the system under
development is rarely, if ever, known at the design stage.

If it is not adequately accounted for, the aforementioned uncertainty in
computer systems leads to degradation of the quality of service in the best case
and to severe faults and burnt devices in the worst-case scenario.
