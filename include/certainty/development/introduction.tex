Power consumption and heat dissipation are of great importance. Power translates
to energy, and energy to hours of battery life and to electricity bills.
Temperature, on the other hand, is one of the major causes of permanent damage
\cite{jedec2016}, which necessitates adequate cooling equipment and, hence,
escalates product expenses \cite{chaudhry2015}. The situation is deteriorated
further by the power-temperature interplay: higher power leads to higher
temperature, and higher temperature to higher power \cite{liu2007}. Therefore,
accounting for power and temperature is key to achieving effectiveness,
efficiency, and robustness.

With the above concern in mind, our interest revolves primarily around analyzing
power and temperature. In this context, it is important to realize that power
analysis is inseparable from temperature analysis due to the aforementioned
power-temperature interplay, which is to be discussed further in
\sref{power-model} and \sref{power-temperature-interplay}. Therefore, in what
follows, power analysis is always implied whenever we discuss temperature
analysis, and vice versa.

There are two types of temperature analysis: \one~transient analysis and
\two~steady-state analysis. The first delivers the temperatures that the system
transitions over within an arbitrary time interval staring from an arbitrary
initial condition when it is exposed to an arbitrary, both spatially and
temporally, power distribution. The second delivers the temperature that the
system attains and retains once it has reached a thermal equilibrium under a
power distribution that is spatially arbitrary but temporally constant or
repeating.

Steady-state analysis can be further classified into two categories: \one~static
analysis and \two~dynamic analysis. The first is concerned with temporally
constant power distributions, that is, with those that do not change over time.
In this case, the resulting temperature distributions do not change over time
either. The second is concerned with temporary arbitrary power distributions
that are periodic, that is, with those that repeat with certain periods. In this
case, the resulting temperature distributions change over time with the same
periods as the corresponding repeating power distributions. The considered
scenario is that the system at hand is exposed to a periodic workload or to such
a workload that can be approximated as periodic. Prominent examples that have
such periodic behaviors are various multimedia applications.

The agenda for this chapter is as follows. In \sref{system-model} and
\sref{power-model}, the system and power models, respectively, are described. In
\sref{temperature-model}, we present the general temperature model that is used
in the above three types of temperature analysis. Transient analysis, static
steady-state analysis, and dynamic steady-state analysis are discussed
separately in \sref{transient-analysis}, \sref{static-steady-state-analysis},
and \sref{dynamic-steady-state-analysis}, respectively. The interdependence
between power and temperature is elaborated on in
\sref{power-temperature-interplay}. Reliability analysis is introduced in
\sref{reliability-analysis}. The importance of temperature analysis for
design-space exploration is covered in \sref{reliability-optimization}.
\sref{utopia-conclusion} concludes the chapter.
