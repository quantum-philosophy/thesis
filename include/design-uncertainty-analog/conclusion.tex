In this chapter, we have presented a framework targeted at the quantification of
transient power and temperature variations of electronic systems under process
variation. The framework is capable of modeling diverse probability laws of the
underlying uncertain parameters and arbitrary dependencies of the system on such
parameters. For a given system under a given workload, our technique delivers
analytical representations of the corresponding stochastic power and temperature
profiles. These representations allow for a computationally efficient estimation
of the probability distributions and accompanying quantities of the power and
temperature characteristics of the system.

Our general technique has been applied in a context of particular importance
where the variability of the effective channel length has been addressed. Note,
however, that the framework can be readily utilized to analyze any other
quantities affected by process variation and to study their combinations. Using
this application, we have drawn a comparison with \ac{MC} sampling, which has
confirmed the efficiency of our approach in terms of both accuracy and speed.
The reduced execution times, by up to five orders of magnitude, implied by our
framework allow for the analysis to be efficiently performed inside design space
exploration loops aimed at, for instance, energy and reliability optimization
with temperature-related constraints under process variation.

We have presented a number of techniques for uncertainty quantification of
electronic systems subjected to process variation. First, we developed a
process-variation-aware approach to dynamic steady-state temperature analysis.
Second, we proposed a framework for reliability analysis that seamlessly takes
into account the variability of process parameters and, in particular, the
effect of process variation on temperature. We drew a comparison with \ac{MC}
sampling, which confirmed the efficiency of our solutions in terms of both
accuracy and speed. The low computational demand of our techniques implies that
they are readily applicable for practical instantiations inside design-space
exploration loops, which was also demonstrated in this work considering an
energy-driven probabilistic optimization procedure under reliability-related
constraints. We have shown that temperature is to be treated as a stochastic
quantity in order to pursue robustness of electronic system designs.

Electronic system designs that ignore process variation are unreliable and
inefficient. In this work, we propose a system-level framework for the analysis
of temperature-induced failures that takes into account the uncertainty due to
process variation. As an intermediate step, we also develop a probabilistic
technique for dynamic steady-state temperature analysis. Given an electronic
system under a certain workload, our framework delivers the corresponding
survival function, founded on the basis of well-established reliability models,
with a closed-form stochastic parameterization in terms of the quantities that
are uncertain at the design stage. The proposed solution is exemplified
considering systems with periodic workloads that suffer from the thermal-cycling
fatigue. The analysis of this fatigue is a challenging problem as it requires
the availability of detailed temperature profiles, which are uncertain due to
the variability of process parameters. In order to demonstrate the computational
efficiency of our framework, we undertake a design-space exploration procedure
to minimize the expected energy consumption under a set of timing, thermal, and
reliability constraints.
